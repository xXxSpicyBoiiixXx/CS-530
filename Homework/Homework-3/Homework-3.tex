\documentclass[12pt]{article}
 
\usepackage[margin=1in]{geometry} 
\usepackage{amsmath,amsthm,amssymb}
\usepackage{mathtools}
\DeclarePairedDelimiter{\ceil}{\lceil}{\rceil}
 
\newcommand{\N}{\mathbb{N}}
\newcommand{\Z}{\mathbb{Z}}
 
\newenvironment{theorem}[2][Theorem]{\begin{trivlist}
\item[\hskip \labelsep {\bfseries #1}\hskip \labelsep {\bfseries #2.}]}{\end{trivlist}}
\newenvironment{lemma}[2][Lemma]{\begin{trivlist}
\item[\hskip \labelsep {\bfseries #1}\hskip \labelsep {\bfseries #2.}]}{\end{trivlist}}
\newenvironment{exercise}[2][Exercise]{\begin{trivlist}
\item[\hskip \labelsep {\bfseries #1}\hskip \labelsep {\bfseries #2.}]}{\end{trivlist}}
\newenvironment{reflection}[2][Reflection]{\begin{trivlist}
\item[\hskip \labelsep {\bfseries #1}\hskip \labelsep {\bfseries #2.}]}{\end{trivlist}}
\newenvironment{proposition}[2][Proposition]{\begin{trivlist}
\item[\hskip \labelsep {\bfseries #1}\hskip \labelsep {\bfseries #2.}]}{\end{trivlist}}
\newenvironment{corollary}[2][Corollary]{\begin{trivlist}
\item[\hskip \labelsep {\bfseries #1}\hskip \labelsep {\bfseries #2.}]}{\end{trivlist}}
 
\begin{document}
 
\title{Homework 3}
\author{Md Ali \\ 
CS 530: Theory of Computation} 
\date{March 5, 2021}

\maketitle
 
\begin{exercise}{1}
Complete the proof of Lemma 1 and Lemma 2 of Lecture 12.
\end{exercise} 

\begin{proof}

\end{proof} 

\begin{exercise}{2}
For each of the following languages, design a context-free grammar that generates the language. Prove the correctness of your construction. \\ \\ 
a. The set of all balanced parenthesis where every opening parenthesis has a corresponding closing one, the alphabet being $\Sigma = \{(,)\}$. Nested parentehsis are allowed. \\ \\ 
b. $\{w: w = 0^{n}1^{m} | n \neq 3m, \Sigma = \{0, 1\} \}$
\end{exercise}
 
\begin{proof}

\end{proof}

\begin{exercise}{3}

\end{exercise}

\begin{proof}
Let G be the following grammer: \\ \\ 
$S \rightarrow aB | bA$ \\ 
$A \rightarrow a | aS | bAA$ \\ 
$B \rightarrow b | bS | aBB$ \\ \\ 
For a string $abaaabbabbba$, find the leftmost derivation and a parse tree. 
\end{proof}

\begin{exercise}{4}
If $L_{1} \subseteq \Sigma^{*}$ and $L_{2} \subseteq \Sigma^{*}$ are languages, the right quotient of $L_{1}$ by $L_{2}$ is defined as follows. \\ \\ 
$L_{1} | L_{2} = \{w \in \Sigma^{*}:$ there is a $u \in L_{w}$ such that $wu \in L_{1}\}$ \\ \\ 
Show that if $L_{1}$ is context-free and $R$ is regular, then $L_{1} | R$ is context-free.
\end{exercise}

\begin{proof}

\end{proof}

\begin{exercise}{5}
Are the following languages context free or not: \\ \\ 
a. $\{a^{p} | p$ is prime \} \\ \\ 
b. $\{a^{n^{2}} : n \geq 0\}$ \\ \\
c. $\{www: w \in (a \cup b)^{*} \}$
\end{exercise}

\begin{proof}

\end{proof}

\end{document}