\documentclass[12pt]{article}
 
\usepackage[margin=1in]{geometry} 
\usepackage{amsmath,amsthm,amssymb}
\usepackage{mathtools}
\DeclarePairedDelimiter{\ceil}{\lceil}{\rceil}
 
\newcommand{\N}{\mathbb{N}}
\newcommand{\Z}{\mathbb{Z}}
 
\newenvironment{theorem}[2][Theorem]{\begin{trivlist}
\item[\hskip \labelsep {\bfseries #1}\hskip \labelsep {\bfseries #2.}]}{\end{trivlist}}
\newenvironment{lemma}[2][Lemma]{\begin{trivlist}
\item[\hskip \labelsep {\bfseries #1}\hskip \labelsep {\bfseries #2.}]}{\end{trivlist}}
\newenvironment{exercise}[2][Exercise]{\begin{trivlist}
\item[\hskip \labelsep {\bfseries #1}\hskip \labelsep {\bfseries #2.}]}{\end{trivlist}}
\newenvironment{reflection}[2][Reflection]{\begin{trivlist}
\item[\hskip \labelsep {\bfseries #1}\hskip \labelsep {\bfseries #2.}]}{\end{trivlist}}
\newenvironment{proposition}[2][Proposition]{\begin{trivlist}
\item[\hskip \labelsep {\bfseries #1}\hskip \labelsep {\bfseries #2.}]}{\end{trivlist}}
\newenvironment{corollary}[2][Corollary]{\begin{trivlist}
\item[\hskip \labelsep {\bfseries #1}\hskip \labelsep {\bfseries #2.}]}{\end{trivlist}}
 
\begin{document}
 
\title{Homework 0}
\author{Md Ali \\ 
CS 530: Theory of Computation} 
\date{January 29, 2021}

\maketitle
 
\begin{exercise}{1}
Show that the set of reals between 0 and 1, in the interval $(0,1)$ is uncountable.
\end{exercise} 

\begin{proof}
We will use proof by contradiction, so assume that the set of reals in the interval of $(0,1)$ are indeed countable. \\ \\ 
This makes all the decimal expansions of the reals in $(0,1)$ to be
\begin{align}
    0.a_{11}a_{12}a_{13} ... \nonumber \\
    0.a_{21}a_{22}a_{23} ... \nonumber \\
    0.a_{31}a_{32}a_{33} .... \nonumber \\
    ... \nonumber  \\
    ... \nonumber \\
    ... \nonumber
\end{align}

So taking this list we can define a decimal value $x$, where $x = x_{1}x_{2}x_{3}...$ where 
\begin{align}
    x_{1} \neq a_{11} \text{ (or 9)}\nonumber \\
    x_{2} \neq a_{22} \text{ (or 9)}\nonumber \\
    x_{3} \neq a_{33} \text{ (or 9)}\nonumber \\
    ... \nonumber
\end{align}

This continues on. Then this decimal expansion of $x$ does not end in recurring 9's and it is different from the $n$th element in the $n$th decimal place. This leads to represent an element of the interval of $(0,1)$ in the reals is not in the list, meaning that we do not have a list. This means that we do not have a list of the reals in the interval of $(0,1)$. Hence result, as this is a contradiction proving that the set of reals between $0$ and $1$, in the interval $(0,1)$ is indeed uncountable. 

\end{proof} 

\begin{exercise}{2}
Show the equivalence of the following, an undirected graph has no cycles if and only if there is a unique path between all pairs of vertices.
\end{exercise}
 
\begin{proof}
First We will prove the biconditional with the antecedent first implicating the consequent and then prove the consequent implicating towards the antecedent. \\ \\
We will take the contradiction where we assume an undirected graph has at least one cycle, and we will attempt to prove that there is a unique path between all pairs of vertices. By definition of a cycle, a cycle is a close path, meaning we must start and end at the same vertex. This contradicts the statement of having a unique path as there will always be two separate paths between any pair of vertices due to the definition of a cycle. \\ \\ 
Hence by contradiction, if an undirected graph has no cycles then there is a unique path between all pairs of vertices. \\ \\
Now we must prove the other part of the biconditional, where if there is a unique path between all pair than an undirected graph has no cycles. \\ \\
Here we will also utilized proof by contradition. Here we will assume that there is no unique paths. Hence, if we have a pair of vertices and we have to have no unique paths, we will ultimately construct at least one cycle. \\ \\ 
Hence, by contradiction, there must be a unique path between all pairs then an undirected graph has no cycles. \\ \\
With the above proof, the biconditional is proven. Hence result.

\end{proof}

\begin{exercise}{3}
Look at the diagram in the notes from class 2 and describe the set of strings that lead from start to final state. 
\end{exercise}

\begin{proof}
The formal description will be $(\{q_{1},q_{2}\},\{0,1\},\delta,q_{1},\{q_{2}\})$ Hence this diagram after a few sample input we can see that all strings are accepted as long as it end in $1$.

\end{proof}
 
\begin{exercise}{4}
Prove the following Lemma. Let $w$ be a string that takes machine $M_{a}$ from $q_{0}^{A}$ to $q_{w}^{A}$ and machine $M_{B}$ from $q_{0}^{B}$ to $q_{w}^{B}$, then $w$ takes $M_{C}$ from  $q_{0}^{C}$ to $q_{w}^{C}$, where  $q_{0}^{C} = (q_{0}^{A},q_{0}^{B})$ and $q_{w}^{C} = (q_{w}^{A},q_{w}^{B})$
\end{exercise}

\begin{proof}
We will utilize proof by induction, if $w = \epsilon$ then this is obviously true as the state $q_{0}^{C} = (q_{0}^{A},q_{0}^{B})$ is true. \\ \\
We will assume that for string of length is $\leq k$. We will have to show that the statement is true for $k+1$. So lets take where $w = k$. We see that $M_{a}$ will be $q_{0}^{A}$ to $q_{k}^{A}$ and $M_{B}$ from $q_{0}^{B}$ to $q_{k}^{B}$. Hence, $M_{C}$ from  $q_{0}^{C}$ to $q_{k}^{C}$. This will conclude that $q_{0}^{C} = (q_{0}^{A},q_{0}^{B})$ and $q_{k}^{C} = (q_{k}^{A},q_{k}^{B})$. Now we can take a look at the case where $w = k+1$. \\ \\
So for the case where $w = k+1$, $M_{a}$ will be $q_{0}^{A}$ to $q_{k+1}^{A}$ and $M_{B}$ from $q_{0}^{B}$ to $q_{k+1}^{B}$. Hence, $M_{C}$ from  $q_{0}^{C}$ to $q_{k+1}^{C}$. This will conclude that $q_{0}^{C} = (q_{0}^{A},q_{0}^{B})$ and $q_{k+1}^{C} = (q_{k+1}^{A},q_{k+1}^{B})$ \\ \\
Hence by induction, if $w$ is to be a string that takes machine $M_{a}$ from $q_{0}^{A}$ to $q_{w}^{A}$ and machine $M_{B}$ from $q_{0}^{B}$ to $q_{w}^{B}$, then $w$ takes $M_{C}$ from  $q_{0}^{C}$ to $q_{w}^{C}$, where  $q_{0}^{C} = (q_{0}^{A},q_{0}^{B})$ and $q_{w}^{C} = (q_{w}^{A},q_{w}^{B})$ is indeed true.

\end{proof}

\end{document}
